% Options for packages loaded elsewhere
% Options for packages loaded elsewhere
\PassOptionsToPackage{unicode}{hyperref}
\PassOptionsToPackage{hyphens}{url}
\PassOptionsToPackage{dvipsnames,svgnames,x11names}{xcolor}
%
\documentclass[
]{article}
\usepackage{xcolor}
\usepackage{amsmath,amssymb}
\setcounter{secnumdepth}{-\maxdimen} % remove section numbering
\usepackage{iftex}
\ifPDFTeX
  \usepackage[T1]{fontenc}
  \usepackage[utf8]{inputenc}
  \usepackage{textcomp} % provide euro and other symbols
\else % if luatex or xetex
  \usepackage{unicode-math} % this also loads fontspec
  \defaultfontfeatures{Scale=MatchLowercase}
  \defaultfontfeatures[\rmfamily]{Ligatures=TeX,Scale=1}
\fi
\usepackage{lmodern}
\ifPDFTeX\else
  % xetex/luatex font selection
\fi
% Use upquote if available, for straight quotes in verbatim environments
\IfFileExists{upquote.sty}{\usepackage{upquote}}{}
\IfFileExists{microtype.sty}{% use microtype if available
  \usepackage[]{microtype}
  \UseMicrotypeSet[protrusion]{basicmath} % disable protrusion for tt fonts
}{}
\makeatletter
\@ifundefined{KOMAClassName}{% if non-KOMA class
  \IfFileExists{parskip.sty}{%
    \usepackage{parskip}
  }{% else
    \setlength{\parindent}{0pt}
    \setlength{\parskip}{6pt plus 2pt minus 1pt}}
}{% if KOMA class
  \KOMAoptions{parskip=half}}
\makeatother
% Make \paragraph and \subparagraph free-standing
\makeatletter
\ifx\paragraph\undefined\else
  \let\oldparagraph\paragraph
  \renewcommand{\paragraph}{
    \@ifstar
      \xxxParagraphStar
      \xxxParagraphNoStar
  }
  \newcommand{\xxxParagraphStar}[1]{\oldparagraph*{#1}\mbox{}}
  \newcommand{\xxxParagraphNoStar}[1]{\oldparagraph{#1}\mbox{}}
\fi
\ifx\subparagraph\undefined\else
  \let\oldsubparagraph\subparagraph
  \renewcommand{\subparagraph}{
    \@ifstar
      \xxxSubParagraphStar
      \xxxSubParagraphNoStar
  }
  \newcommand{\xxxSubParagraphStar}[1]{\oldsubparagraph*{#1}\mbox{}}
  \newcommand{\xxxSubParagraphNoStar}[1]{\oldsubparagraph{#1}\mbox{}}
\fi
\makeatother

\usepackage{color}
\usepackage{fancyvrb}
\newcommand{\VerbBar}{|}
\newcommand{\VERB}{\Verb[commandchars=\\\{\}]}
\DefineVerbatimEnvironment{Highlighting}{Verbatim}{commandchars=\\\{\}}
% Add ',fontsize=\small' for more characters per line
\usepackage{framed}
\definecolor{shadecolor}{RGB}{241,243,245}
\newenvironment{Shaded}{\begin{snugshade}}{\end{snugshade}}
\newcommand{\AlertTok}[1]{\textcolor[rgb]{0.68,0.00,0.00}{#1}}
\newcommand{\AnnotationTok}[1]{\textcolor[rgb]{0.37,0.37,0.37}{#1}}
\newcommand{\AttributeTok}[1]{\textcolor[rgb]{0.40,0.45,0.13}{#1}}
\newcommand{\BaseNTok}[1]{\textcolor[rgb]{0.68,0.00,0.00}{#1}}
\newcommand{\BuiltInTok}[1]{\textcolor[rgb]{0.00,0.23,0.31}{#1}}
\newcommand{\CharTok}[1]{\textcolor[rgb]{0.13,0.47,0.30}{#1}}
\newcommand{\CommentTok}[1]{\textcolor[rgb]{0.37,0.37,0.37}{#1}}
\newcommand{\CommentVarTok}[1]{\textcolor[rgb]{0.37,0.37,0.37}{\textit{#1}}}
\newcommand{\ConstantTok}[1]{\textcolor[rgb]{0.56,0.35,0.01}{#1}}
\newcommand{\ControlFlowTok}[1]{\textcolor[rgb]{0.00,0.23,0.31}{\textbf{#1}}}
\newcommand{\DataTypeTok}[1]{\textcolor[rgb]{0.68,0.00,0.00}{#1}}
\newcommand{\DecValTok}[1]{\textcolor[rgb]{0.68,0.00,0.00}{#1}}
\newcommand{\DocumentationTok}[1]{\textcolor[rgb]{0.37,0.37,0.37}{\textit{#1}}}
\newcommand{\ErrorTok}[1]{\textcolor[rgb]{0.68,0.00,0.00}{#1}}
\newcommand{\ExtensionTok}[1]{\textcolor[rgb]{0.00,0.23,0.31}{#1}}
\newcommand{\FloatTok}[1]{\textcolor[rgb]{0.68,0.00,0.00}{#1}}
\newcommand{\FunctionTok}[1]{\textcolor[rgb]{0.28,0.35,0.67}{#1}}
\newcommand{\ImportTok}[1]{\textcolor[rgb]{0.00,0.46,0.62}{#1}}
\newcommand{\InformationTok}[1]{\textcolor[rgb]{0.37,0.37,0.37}{#1}}
\newcommand{\KeywordTok}[1]{\textcolor[rgb]{0.00,0.23,0.31}{\textbf{#1}}}
\newcommand{\NormalTok}[1]{\textcolor[rgb]{0.00,0.23,0.31}{#1}}
\newcommand{\OperatorTok}[1]{\textcolor[rgb]{0.37,0.37,0.37}{#1}}
\newcommand{\OtherTok}[1]{\textcolor[rgb]{0.00,0.23,0.31}{#1}}
\newcommand{\PreprocessorTok}[1]{\textcolor[rgb]{0.68,0.00,0.00}{#1}}
\newcommand{\RegionMarkerTok}[1]{\textcolor[rgb]{0.00,0.23,0.31}{#1}}
\newcommand{\SpecialCharTok}[1]{\textcolor[rgb]{0.37,0.37,0.37}{#1}}
\newcommand{\SpecialStringTok}[1]{\textcolor[rgb]{0.13,0.47,0.30}{#1}}
\newcommand{\StringTok}[1]{\textcolor[rgb]{0.13,0.47,0.30}{#1}}
\newcommand{\VariableTok}[1]{\textcolor[rgb]{0.07,0.07,0.07}{#1}}
\newcommand{\VerbatimStringTok}[1]{\textcolor[rgb]{0.13,0.47,0.30}{#1}}
\newcommand{\WarningTok}[1]{\textcolor[rgb]{0.37,0.37,0.37}{\textit{#1}}}

\usepackage{longtable,booktabs,array}
\usepackage{calc} % for calculating minipage widths
% Correct order of tables after \paragraph or \subparagraph
\usepackage{etoolbox}
\makeatletter
\patchcmd\longtable{\par}{\if@noskipsec\mbox{}\fi\par}{}{}
\makeatother
% Allow footnotes in longtable head/foot
\IfFileExists{footnotehyper.sty}{\usepackage{footnotehyper}}{\usepackage{footnote}}
\makesavenoteenv{longtable}
\usepackage{graphicx}
\makeatletter
\newsavebox\pandoc@box
\newcommand*\pandocbounded[1]{% scales image to fit in text height/width
  \sbox\pandoc@box{#1}%
  \Gscale@div\@tempa{\textheight}{\dimexpr\ht\pandoc@box+\dp\pandoc@box\relax}%
  \Gscale@div\@tempb{\linewidth}{\wd\pandoc@box}%
  \ifdim\@tempb\p@<\@tempa\p@\let\@tempa\@tempb\fi% select the smaller of both
  \ifdim\@tempa\p@<\p@\scalebox{\@tempa}{\usebox\pandoc@box}%
  \else\usebox{\pandoc@box}%
  \fi%
}
% Set default figure placement to htbp
\def\fps@figure{htbp}
\makeatother





\setlength{\emergencystretch}{3em} % prevent overfull lines

\providecommand{\tightlist}{%
  \setlength{\itemsep}{0pt}\setlength{\parskip}{0pt}}



 


\usepackage{xcolor}
\definecolor{mylink}{HTML}{798B93}
\makeatletter
\@ifpackageloaded{tcolorbox}{}{\usepackage[skins,breakable]{tcolorbox}}
\@ifpackageloaded{fontawesome5}{}{\usepackage{fontawesome5}}
\definecolor{quarto-callout-color}{HTML}{909090}
\definecolor{quarto-callout-note-color}{HTML}{0758E5}
\definecolor{quarto-callout-important-color}{HTML}{CC1914}
\definecolor{quarto-callout-warning-color}{HTML}{EB9113}
\definecolor{quarto-callout-tip-color}{HTML}{00A047}
\definecolor{quarto-callout-caution-color}{HTML}{FC5300}
\definecolor{quarto-callout-color-frame}{HTML}{acacac}
\definecolor{quarto-callout-note-color-frame}{HTML}{4582ec}
\definecolor{quarto-callout-important-color-frame}{HTML}{d9534f}
\definecolor{quarto-callout-warning-color-frame}{HTML}{f0ad4e}
\definecolor{quarto-callout-tip-color-frame}{HTML}{02b875}
\definecolor{quarto-callout-caution-color-frame}{HTML}{fd7e14}
\makeatother
\makeatletter
\@ifpackageloaded{caption}{}{\usepackage{caption}}
\AtBeginDocument{%
\ifdefined\contentsname
  \renewcommand*\contentsname{Table of contents}
\else
  \newcommand\contentsname{Table of contents}
\fi
\ifdefined\listfigurename
  \renewcommand*\listfigurename{List of Figures}
\else
  \newcommand\listfigurename{List of Figures}
\fi
\ifdefined\listtablename
  \renewcommand*\listtablename{List of Tables}
\else
  \newcommand\listtablename{List of Tables}
\fi
\ifdefined\figurename
  \renewcommand*\figurename{Figure}
\else
  \newcommand\figurename{Figure}
\fi
\ifdefined\tablename
  \renewcommand*\tablename{Table}
\else
  \newcommand\tablename{Table}
\fi
}
\@ifpackageloaded{float}{}{\usepackage{float}}
\floatstyle{ruled}
\@ifundefined{c@chapter}{\newfloat{codelisting}{h}{lop}}{\newfloat{codelisting}{h}{lop}[chapter]}
\floatname{codelisting}{Listing}
\newcommand*\listoflistings{\listof{codelisting}{List of Listings}}
\makeatother
\makeatletter
\makeatother
\makeatletter
\@ifpackageloaded{caption}{}{\usepackage{caption}}
\@ifpackageloaded{subcaption}{}{\usepackage{subcaption}}
\makeatother
\usepackage{bookmark}
\IfFileExists{xurl.sty}{\usepackage{xurl}}{} % add URL line breaks if available
\urlstyle{same}
\hypersetup{
  pdftitle={Lab 3},
  colorlinks=true,
  linkcolor={mylink},
  filecolor={Maroon},
  citecolor={mylink},
  urlcolor={mylink},
  pdfcreator={LaTeX via pandoc}}


\title{Lab 3}
\usepackage{etoolbox}
\makeatletter
\providecommand{\subtitle}[1]{% add subtitle to \maketitle
  \apptocmd{\@title}{\par {\large #1 \par}}{}{}
}
\makeatother
\subtitle{Due Thursday September 18 at 11:59 PM}
\author{}
\date{}
\begin{document}
\maketitle


\begin{tcolorbox}[enhanced jigsaw, colback=white, colframe=quarto-callout-note-color-frame, coltitle=black, titlerule=0mm, opacityback=0, bottomtitle=1mm, left=2mm, colbacktitle=quarto-callout-note-color!10!white, toptitle=1mm, breakable, toprule=.15mm, arc=.35mm, title=\textcolor{quarto-callout-note-color}{\faInfo}\hspace{0.5em}{Note}, bottomrule=.15mm, rightrule=.15mm, leftrule=.75mm, opacitybacktitle=0.6]

This assignment was adapted from:

\begin{itemize}
\tightlist
\item
  Meng, Xiao-Li (2023):
  ``\href{https://doi.org/10.51387/22-NEJSDS6}{Double your variance,
  dirtify your Bayes, devour your pufferfish, and draw your
  kidstrogram},'' \emph{The New England Journal of Statistics in Data
  Science}, vol 1 no 1.
\end{itemize}

\end{tcolorbox}

\emph{Car Talk} (1977 - 2012) was a popular show on NPR. It featured a
segment called the ``Puzzler,'' where the hosts would read a brain
teaser and invite listeners to send in solutions for the chance to win a
prize. Here is the text of one such Puzzler (original audio at around
19:19
\href{https://www.npr.org/2015/07/11/422108689/-1528-forget-the-car-find-a-convent}{here}):

\begin{quote}
There's a rare disease that's sweeping through your town. Of all the
people who are exposed to it, 0.1\% of the people actually contract the
disease. There are no symptoms until the disease actually occurs.
However, there's a diagnostic test that can detect the presence of the
disease up to a year before it strikes. You go to your doctor, and he
administers the test. It comes out positive. You say, ``I'm done for!''
Then you get a little bit encouraged. You say, ``Wait a minute, doc, is
this test 100\% accurate?'' Your doctor responds, ``Well, not really.
It's 95\% accurate.'' In other words, 5\% of the people who take the
test will test positive but they don't really have the disease.
\textbf{What are the chances that you actually have the disease?}
\end{quote}

In what follows, you may find some notation useful. Let \(D\) denote
your true disease status, and \(T\) denote the result of your test. Then

\[
\begin{align*}
    p &= P(D=+) && \text{(prevalence)}
    \\
    f_{-}&=P(T=-\,|\, D=+) && \text{(false negative rate)}
    \\
    f_{+}&=P(T=+\,|\, D=-) && \text{(false positive rate)}
    \\
    1-f_{-}&=P(T=+\,|\, D = +) && \text{(sensitivity)}
    \\
    1-f_{+}&=P(T=-\,|\, D = -) && \text{(specificity)}.
\end{align*}
\]

\subsection{Task 1}\label{task-1}

Before you read anything else below, try to solve the puzzler yourself.
What do you come up with?

\begin{tcolorbox}[enhanced jigsaw, colback=white, colframe=quarto-callout-warning-color-frame, coltitle=black, titlerule=0mm, opacityback=0, bottomtitle=1mm, left=2mm, colbacktitle=quarto-callout-warning-color!10!white, toptitle=1mm, breakable, toprule=.15mm, arc=.35mm, title=\textcolor{quarto-callout-warning-color}{\faExclamationTriangle}\hspace{0.5em}{Solution}, bottomrule=.15mm, rightrule=.15mm, leftrule=.75mm, opacitybacktitle=0.6]

Obviously you will struggle to come up with anything, and a lot depends
on how you interpret ``It's 95\% accurate\ldots.5\% of the people who
take the test will test positive but they don't really have the
disease.'' My solution below interprets this as the false positive rate.
The puzzler says nothing about the false negative rate, and so you can't
actually answer the question because you do not have enough information.

\end{tcolorbox}

\subsection{Task 2}\label{task-2}

Explain the correct way to solve this problem. Exactly what probability
are we asked to compute, and what formula should we apply to do it? Does
the prompt actually provide enough information to ultimately get the job
done?

\begin{tcolorbox}[enhanced jigsaw, colback=white, colframe=quarto-callout-warning-color-frame, coltitle=black, titlerule=0mm, opacityback=0, bottomtitle=1mm, left=2mm, colbacktitle=quarto-callout-warning-color!10!white, toptitle=1mm, breakable, toprule=.15mm, arc=.35mm, title=\textcolor{quarto-callout-warning-color}{\faExclamationTriangle}\hspace{0.5em}{Solution}, bottomrule=.15mm, rightrule=.15mm, leftrule=.75mm, opacitybacktitle=0.6]

The Puzzler asks us to compute \(P(D=+\,|\, T=+)\), which in principle
we can do with Bayes' theorem:

\[
\begin{aligned}
    P(D=+\,|\, T=+)
    &=
    \frac{P(T=+\,|\, D=+)P(D=+)}{P(T=+)}
    \\
    &=
    \frac{P(T=+\,|\, D=+)P(D=+)}{P(T=+\,|\, D=+)P(D=+)+P(T=+\,|\, D=-)P(D=-)}
    \\
    &=
    \frac{(1-f_-)p}{(1-f_-)p+f_+(1-p)}
    .
\end{aligned}
\]

Unfortunately, the show doesn't tell us \(f_-\), so you cannot actually
compute the answer.

\end{tcolorbox}

\subsection{Task 3}\label{task-3}

Beneath the fold is the solution that the hosts ultimately revealed.

\begin{tcolorbox}[enhanced jigsaw, colback=white, colframe=quarto-callout-tip-color-frame, coltitle=black, titlerule=0mm, opacityback=0, bottomtitle=1mm, left=2mm, colbacktitle=quarto-callout-tip-color!10!white, toptitle=1mm, breakable, toprule=.15mm, arc=.35mm, title=\textcolor{quarto-callout-tip-color}{\faLightbulb}\hspace{0.5em}{The show's solution}, bottomrule=.15mm, rightrule=.15mm, leftrule=.75mm, opacitybacktitle=0.6]

Original audio at around 20:30
\href{https://www.npr.org/2015/07/11/422108689/-1528-forget-the-car-find-a-convent}{here}:

\begin{quote}
Let's say 1000 people take the test. Fifty people will test positive and
yet they will not have it. One will test positive and have it. So your
chances of actually having it, even though you tested positive, are one
in 51, or a little less than 2\%.
\end{quote}

\end{tcolorbox}

So, what do you think? Did they get it right? Explain how the show
interpreted the information given in the Puzzler, and explain what
arithmetic formula they implicitly applied in order to compute their
answer.

\begin{tcolorbox}[enhanced jigsaw, colback=white, colframe=quarto-callout-warning-color-frame, coltitle=black, titlerule=0mm, opacityback=0, bottomtitle=1mm, left=2mm, colbacktitle=quarto-callout-warning-color!10!white, toptitle=1mm, breakable, toprule=.15mm, arc=.35mm, title=\textcolor{quarto-callout-warning-color}{\faExclamationTriangle}\hspace{0.5em}{Solution}, bottomrule=.15mm, rightrule=.15mm, leftrule=.75mm, opacitybacktitle=0.6]

When they say ``0.1\% of the people actually contract the disease,''
this refers to the prevalence, and when they say ``5\% of the people who
take the test will test positive but they don't really have the
disease,'' this refers to the false positive rate. So \(p=0.001\), and
\(f_+=0.05\). Based only on these, they somehow came up with:

\[
\frac{p}{p+f_+}=\frac{0.001}{.001 + 0.05} = \frac{1}{51}\approx 0.0196.
\]

\end{tcolorbox}

\subsection{Task 4}\label{task-4}

Explain two things:

\begin{enumerate}
\def\labelenumi{\alph{enumi}.}
\tightlist
\item
  Under what conditions is the formula that the show used actually an
  \emph{upper bound} on the correct answer?
\item
  Even though their number is not exactly correct, why might an upper
  bound on the true probability still be a useful thing to calculate?
\end{enumerate}

\begin{tcolorbox}[enhanced jigsaw, colback=white, colframe=quarto-callout-warning-color-frame, coltitle=black, titlerule=0mm, opacityback=0, bottomtitle=1mm, left=2mm, colbacktitle=quarto-callout-warning-color!10!white, toptitle=1mm, breakable, toprule=.15mm, arc=.35mm, title=\textcolor{quarto-callout-warning-color}{\faExclamationTriangle}\hspace{0.5em}{Solution}, bottomrule=.15mm, rightrule=.15mm, leftrule=.75mm, opacitybacktitle=0.6]

By factoring \(1-f_-\) out of the numerator and denominator, we can
rewrite the correct formula as

\[
P(D=+\,|\, T = +)=\frac{(1-f_-)p}{(1-f_-)p+f_+(1-p)}=\frac{p}{p+f_+\left(\frac{1-p}{1-f_-}\right)}.
\]

If \(p\leq f_-\), then \(1-f_-\leq 1-p\), which implies

\[
p+f_+\leq p+f_+\left(\frac{1-p}{1-f_-}\right).
\]

As a consequence, we see that

\[
P(D=+\,|\, T = +)
=
\frac{p}{p+f_+\left(\frac{1-p}{1-f_-}\right)}
\leq \frac{p}{p+f_+},
\quad \text{when }p\leq f_-
.
\]

So the formula that the show applied is an upper bound for the true
probability.

Two types of errors are possible in a situation like this: treating a
disease that is not there, or failing to treat a disease that is there.
On balance, the second error is probably the worse of the two, so I
would rather overestimate the probability that I have the disease than
underestimate it.

\end{tcolorbox}

\subsection{Task 5}\label{task-5}

So, the show's answer, while wrong, could still be potentially useful
upper bound on the true probability. Neat! But how wrong is it, even? To
get a sense of this,
\href{https://anyakatsevich.github.io/sta240-s26.github.io/explainers/lineplot.html}{use
\texttt{R}} to create some line plots with the prevalence
\(p\in[0,\, 1]\) on the horizontal axis and the true and approximate
probabilities for different values of \(f_-\) and \(f_+\) on the
vertical axis. Mix-and-match \(f_-,\, f_+\in\{0.1,\, 0.2\}\), and
comment on the difference between the two curves.

\begin{tcolorbox}[enhanced jigsaw, colback=white, colframe=quarto-callout-warning-color-frame, coltitle=black, titlerule=0mm, opacityback=0, bottomtitle=1mm, left=2mm, colbacktitle=quarto-callout-warning-color!10!white, toptitle=1mm, breakable, toprule=.15mm, arc=.35mm, title=\textcolor{quarto-callout-warning-color}{\faExclamationTriangle}\hspace{0.5em}{Solution}, bottomrule=.15mm, rightrule=.15mm, leftrule=.75mm, opacitybacktitle=0.6]

\begin{Shaded}
\begin{Highlighting}[]
\NormalTok{FPR }\OtherTok{=} \FunctionTok{c}\NormalTok{(}\FloatTok{0.1}\NormalTok{, }\FloatTok{0.2}\NormalTok{)}
\NormalTok{FNR }\OtherTok{=} \FunctionTok{c}\NormalTok{(}\FloatTok{0.1}\NormalTok{, }\FloatTok{0.2}\NormalTok{)}
\FunctionTok{par}\NormalTok{(}\AttributeTok{mfrow =} \FunctionTok{c}\NormalTok{(}\DecValTok{2}\NormalTok{, }\DecValTok{2}\NormalTok{))}
\ControlFlowTok{for}\NormalTok{ (i }\ControlFlowTok{in} \DecValTok{1}\SpecialCharTok{:}\DecValTok{2}\NormalTok{)\{}
  \ControlFlowTok{for}\NormalTok{ (j }\ControlFlowTok{in} \DecValTok{1}\SpecialCharTok{:}\DecValTok{2}\NormalTok{)\{}
\NormalTok{    f\_plus }\OtherTok{=}\NormalTok{ FPR[i]}
\NormalTok{    f\_minus }\OtherTok{=}\NormalTok{ FNR[j]}
    \FunctionTok{curve}\NormalTok{(x }\SpecialCharTok{/}\NormalTok{ (x }\SpecialCharTok{+}\NormalTok{ f\_plus),}
          \AttributeTok{col =} \StringTok{"blue"}\NormalTok{,}
          \AttributeTok{xlab =} \StringTok{"p"}\NormalTok{, }
          \AttributeTok{ylab =} \StringTok{""}\NormalTok{, }
          \AttributeTok{xlim =} \FunctionTok{c}\NormalTok{(}\DecValTok{0}\NormalTok{, }\DecValTok{1}\NormalTok{),}
          \AttributeTok{ylim =} \FunctionTok{c}\NormalTok{(}\DecValTok{0}\NormalTok{, }\DecValTok{1}\NormalTok{),}
          \AttributeTok{main =} \FunctionTok{paste}\NormalTok{(}\StringTok{"FNR = "}\NormalTok{, f\_minus, }\StringTok{"; FPR = "}\NormalTok{, f\_plus))}
    \FunctionTok{curve}\NormalTok{(x }\SpecialCharTok{/}\NormalTok{ (x }\SpecialCharTok{+}\NormalTok{ f\_plus }\SpecialCharTok{*}\NormalTok{ ((}\DecValTok{1} \SpecialCharTok{{-}}\NormalTok{ x) }\SpecialCharTok{/}\NormalTok{ (}\DecValTok{1} \SpecialCharTok{{-}}\NormalTok{ f\_minus))),}
          \AttributeTok{col =} \StringTok{"red"}\NormalTok{, }\AttributeTok{add =} \ConstantTok{TRUE}\NormalTok{)}
    \FunctionTok{legend}\NormalTok{(}\StringTok{"bottomright"}\NormalTok{,}
           \FunctionTok{c}\NormalTok{(}\StringTok{"True Probability"}\NormalTok{, }\StringTok{"Car Talk Answer"}\NormalTok{),}
           \AttributeTok{col =} \FunctionTok{c}\NormalTok{(}\StringTok{"red"}\NormalTok{, }\StringTok{"blue"}\NormalTok{),}
           \AttributeTok{bty =} \StringTok{"n"}\NormalTok{,}
           \AttributeTok{lty =} \FunctionTok{c}\NormalTok{(}\DecValTok{1}\NormalTok{, }\DecValTok{1}\NormalTok{)}
\NormalTok{           )}
\NormalTok{  \}}
\NormalTok{\}}
\end{Highlighting}
\end{Shaded}

\pandocbounded{\includegraphics[keepaspectratio]{lab-3_files/figure-pdf/unnamed-chunk-1-1.pdf}}

When the prevalence is low (ie rare disease) the Car Talk curve is a
very close upper bound on the true probability. For higher prevalence,
the Car Talk curve is no longer an upper bound, and the curves get
farther apart.

\end{tcolorbox}




\end{document}
